% !TeX spellcheck = en_US
\documentclass[11pt, a4paper]{article}

% Set the title of the current document to be produced.
\newcommand{\doctitle}{Programming Assignment I}
% Command for the due date of the homework.
\newcommand{\duedate}{\color{rltred}{\faCalendarCheckO { }Due date: May 1st, before midnight \faCalendarCheckO	}}

%------------------------------------------------------------
% Import commands for both teacher and course information.  | 
% NOTE: Change your teacher and course info in these files. |
%------>------>------>------>------>------>------>------>-->|
%-------------------------------------------------
% Teacher-specific commands                      |
%---------------                                 |
%-> Instructions: change your teacher info here. |
%------->------>------>------>------>------>---->|
%
\newcommand{\instructor}{Instructor Name}
\newcommand{\office}{Office number}
\newcommand{\hours}{By appointment}
\newcommand{\phone}{514.999.9999}
\newcommand{\college}{College Name}
\newcommand{\email}{username@mycollege.ca}
\newcommand{\faculty}{Faculty of Science and Technology}
\newcommand{\department}{Computer Science Technology}
                              %|
%-------------------------------------------------
% Course-specific commands                       |
%---------------                                 |
%-> Instructions: change your course info here.  |
%------->------>------>------>------>------>---->|
%
\newcommand{\semester}{Winter 2022}
\newcommand{\csection}{00001 \& 00002}
\newcommand{\ponderation}{2-4-3 (Theory-Lab-Homework)}
\newcommand{\coursetitle}{Course Title}
\newcommand{\coursenumber}{[Course Number]}
\newcommand{\prerequisite}{All porgram courses semesters 1-4}
                               %|   
%
%------------------------------------------------------------
%-- Import packages and custom command definitons.          |
%------>------>------>------>------>------>------>------>-->|
%----------------------------------------------------
% The following is a list of LaTeX packages imports |
%------->------>------>------>------>------>---->---|
%

% The margins at the bottom of the page has been reduced.
% this allows for a slim footer.
\usepackage[left=1in,right=1in,top=1in,bottom=0.7in]{geometry}
% Original size:
%\usepackage[inner=1.5cm,outer=1.5cm,top=1.5cm,bottom=.5cm,margin=1in]{geometry}
\usepackage[
    colorlinks,
    pagebackref,
    pdfusetitle,
    urlcolor=blue,
    citecolor=blue,
    linkcolor=blue,    
    plainpages=false]
{hyperref}            
% ftp://ftp.dante.de/tex-archive/fonts/bbding/bbding.pdf
%https://ctan.math.illinois.edu/fonts/bbding/bbding.pdf
\usepackage{fancyhdr, lastpage, bbding, pmboxdraw}
\usepackage{fancyvrb}
\PassOptionsToPackage{usenames,dvipsnames}{xcolor}
\usepackage{acronym}
\usepackage{amsthm}
\usepackage{caption}
\usepackage{xcolor}
\usepackage{enumitem}
\usepackage{tabularx}
\usepackage{sectsty}
% pifont package doc at: https://ctan.math.ca/tex-archive/macros/latex/required/psnfss/psnfss2e.pdf
% pifont is used to define custom list and style list items using the \ding command. 
\usepackage{pifont} 
% bclogo used for making a colored box for notes. 
% @see: https://ctan.org/pkg/bclogo?lang=en
\usepackage[tikz]{bclogo} 
\usepackage{titlesec}  
\usepackage[open,openlevel=1]{bookmark}

%-- @see http://ctan.sharelatex.com/tex-archive/fonts/fontawesome/doc/fontawesome.pdf
% Font Awesome  http://ctan.math.washington.edu/tex-archive/fonts/fontawesome5/doc/fontawesome5.pdf
% https://muug.ca/mirror/ctan/fonts/fontawesome5/doc/fontawesome5.pdf
\usepackage{fontawesome5}
\usepackage{fontawesome}
%---------------------------------
% ==== Font setup.
% Load any of the following fonts.
%---------------------------------
%\usepackage{lmodern}
%\usepackage{mathptmx}
%\usepackage{times}
%\usepackage[sc]{mathpazo} % Palatino font.
%\linespread{1.05} % Palatino needs more leading (space between lines)
\usepackage{tgbonum} % For Bonum/Bookman font.
\usepackage[utf8]{inputenc}
\usepackage[T1]{fontenc}
%---------------------------------
\usepackage{booktabs} 

\pagestyle{empty}
\usepackage{graphicx}
\usepackage{multicol}
\usepackage{blindtext}  
\usepackage{vhistory} % for making a table for the revision history.
                                  %|  
%--------------------------------------------------------
%--> \customhrule: makes a customized rule whose width  | 
%                  should be passed as parameter.       |
%--------------------------------------------------------
\newcommand{\customhrule}[1]{
	\rule[1.4pt]{\linewidth}{#1}
}
%------------------------------------------------------
%--> \doublerule: makes a double rule.                |
%------------------------------------------------------ 
\newcommand{\doublerule}[1][.4pt]{
	\noindent
	\makebox[0pt][l]{\rule[.7ex]{\linewidth}{#1}}%
	\rule[1pt]{\linewidth}{#1}\par} 
%===== Custom Ruler commands  ==================
\renewcommand{\headrulewidth}{1pt}
\renewcommand{\footrulewidth}{0.4pt}

% Disable spaces between list items in a labeled list.
\setlist{noitemsep}
 
%-------------------------------------------------------------
%= The followig are declaraions of custom Lists              =
%-------------------------------------------------------------
%
%======= Green rectangles list =======================
% \Rectangle from bbind
\newlist{greenrectangles}{itemize}{4}
%\setlist[greenrectangles]{topsep=4pt,partopsep=0pt,itemsep=3pt,parsep=0pt,labelindent=0.5cm,leftmargin=*}
\setlist[greenrectangles]{itemsep=5pt,parsep=0pt,topsep=4pt,partopsep=3pt}
\setlist[greenrectangles,1]{font=\color{darkred},label={\color{darkgreen}{\Rectangle}}}

%======= Alphabetical  list =======================
\newlist{alphalist}{enumerate}{9}
\setlist[alphalist]{topsep=4pt,partopsep=0pt,itemsep=3pt,parsep=0pt,labelindent=0.5cm,leftmargin=*}
\setlist[alphalist,1]{label=\textbf{\alph*)}}
%======= Non-numbered list =======================
\newlist{itemizedlist}{itemize}{9}
\setlist[itemizedlist]{topsep=4pt,partopsep=0pt,itemsep=3pt,parsep=0pt,labelindent=0.5cm,leftmargin=*}
%\setlist[itemizedlist,1 ]{label=\textbf{\alph*)}}

%======= Arrowed list =======================
\newlist{arrows}{itemize}{4}
\setlist[arrows]{topsep=4pt,partopsep=0pt,itemsep=3pt,parsep=0pt,labelindent=0.5cm,leftmargin=*}
\setlist[arrows,1]{font=\color{darkred},label={\HandRight}}

%======= Bordered square list =======================
% Colorize the selected symbol? 
% ❏
\newlist{borderedsquare}{itemize}{4}
\setlist[borderedsquare]{topsep=4pt,partopsep=0pt,itemsep=3pt,parsep=0pt,labelindent=0.5cm,leftmargin=*}
\setlist[borderedsquare,1]{label=\ding{111}}

%======= Filled, curved arrow list =======================
\newlist{curveddarrow}{itemize}{4}
\setlist[curveddarrow]{topsep=4pt,partopsep=0pt,itemsep=3pt,parsep=0pt,labelindent=0.5cm,leftmargin=*}
\setlist[curveddarrow,1]{label=\small\faMarker}

%======= Colored pen list ======================= 
\newlist{coloredPen}{itemize}{4}
\setlist[coloredPen]{topsep=4pt,partopsep=0pt,itemsep=3pt,parsep=0pt,labelindent=0.5cm,leftmargin=*}
\setlist[coloredPen,1]{font=\color{darkred},label=\small\faMarker}

%======= Objectives list ======================= 
% ➠
\newlist{objectives}{itemize}{4}
\setlist[objectives]{topsep=4pt,partopsep=0pt,itemsep=3pt,parsep=0pt,labelindent=0.5cm,leftmargin=*}
\setlist[objectives,1]{label=\small\ding{224}}

%======= Dark starred list ======================= 
% ✸
\newlist{filledstarlist}{itemize}{4}
\setlist[filledstarlist]{topsep=4pt,partopsep=0pt,itemsep=3pt,parsep=0pt,labelindent=0.5cm,leftmargin=*}
\setlist[filledstarlist,1]{label=\small\ding{88}}

%======= Dark-bordered empty circle list ======================= 
% ❍
\newlist{emptyCircleList}{itemize}{4}
\setlist[emptyCircleList]{topsep=4pt,partopsep=0pt,itemsep=3pt,parsep=0pt,labelindent=0.5cm,leftmargin=*}
\setlist[emptyCircleList,1]{label=\small\ding{109}}

%======= Filled right arrow list ======================= 
% ➤
\newlist{filledRightArrowList}{itemize}{4}
\setlist[filledRightArrowList]{topsep=4pt,partopsep=0pt,itemsep=3pt,parsep=0pt,labelindent=0.5cm,leftmargin=*}
\setlist[filledRightArrowList,1]{label=\small\ding{228}}

%======= Numbered list: non-filled circle list ======================= 
% ➀
\newlist{numberedEmptyList}{itemize}{9}
\setlist[numberedEmptyList]{topsep=4pt,partopsep=0pt,itemsep=3pt,parsep=0pt,labelindent=0.5cm,leftmargin=*}
\setlist[numberedEmptyList,9]{label=\ding{182}}

%======= Right hand pointing list =======================
\newlist{rightHandPointingList}{itemize}{4}
\setlist[rightHandPointingList]{topsep=4pt,partopsep=0pt,itemsep=3pt,parsep=0pt,labelindent=0.5cm,leftmargin=*}
\setlist[rightHandPointingList,1]{font=\color{darkred},label={\HandRight}}

%----------------------------------------------------------------------
%=   The followig are custom colors declaraions                       |
%--  more colors codes can be found at: http://latexcolor.com/        | 
%-- usage: {\color{declared-color} some text}.                        |    
%  e.g.,: {\color{darkblue}{ This text will appear darkblue-colored}} |
%----------------------------------------------------------------------
\definecolor{darkblue}{rgb}{0,0,.6}
\definecolor{darkred}{rgb}{.7,0,0}
\definecolor{darkgreen}{rgb}{0,.6,0}
\definecolor{darkestred}{rgb}{.8,.1,0}
\definecolor{red}{rgb}{.98,0,0}
\definecolor{OliveGreen}{cmyk}{0.64,0,0.95,0.40}
\definecolor{CadetBlue}{cmyk}{0.62,0.57,0.23,0}
\definecolor{lightlightgray}{gray}{0.93}
\definecolor{vanierred}{RGB}{210,0,2}
\definecolor{darkestblue}{rgb}{0.0, 0.0, 0.55}
\definecolor{darkblue}{rgb}{0,0,.6}
\definecolor{darkred}{rgb}{.7,0,0}
\definecolor{darkgreen}{rgb}{0,.6,0}
\definecolor{darkestred}{rgb}{.8,.1,0}
\definecolor{red}{rgb}{.98,0,0}
\definecolor{OliveGreen}{cmyk}{0.64,0,0.95,0.40}
\definecolor{CadetBlue}{cmyk}{0.62,0.57,0.23,0}
\definecolor{lightlightgray}{gray}{0.93}
\definecolor{darkorange}{rgb}{255,140,0}
\definecolor{fluorescentyellow}{rgb}{0.8, 1.0, 0.0}
\definecolor{darkyellow}{rgb}{1,1,0.34}
\definecolor{lightyellow}{rgb}{1,1,0.6}
\definecolor{coolblack}{rgb}{0.0, 0.18, 0.39}
\definecolor{lightgray}{rgb}{.9,.9,.9}
\definecolor{darkgray}{rgb}{.4,.4,.4}
\definecolor{purple}{rgb}{0.65, 0.12, 0.82}
\definecolor{gray}{rgb}{0.4,0.4,0.4}
\definecolor{cyan}{rgb}{0.0,0.6,0.6}
\definecolor{dkgreen}{rgb}{0,0.6,0}
\definecolor{gray}{rgb}{0.5,0.5,0.5}
\definecolor{mauve}{rgb}{0.58,0,0.82}
\definecolor{lightblue}{rgb}{0.0,0.0,0.9}
\colorlet{punct}{red!60!black}
\definecolor{background}{HTML}{EEEEEE}
\definecolor{delim}{RGB}{20,105,176}
\colorlet{numb}{magenta!60!black}
\definecolor{coolblack}{rgb}{0.0, 0.18, 0.39}
\definecolor{forestgreen}{rgb}{0.0, 0.27, 0.13}
\definecolor{firebrick}{rgb}{0.7, 0.13, 0.13}
\definecolor{rltred}{rgb}{0.75,0,0}
\definecolor{rltgreen}{rgb}{0,0.5,0}
\definecolor{rltblue}{rgb}{0,0,0.75}
\definecolor{indigo}{rgb}{0.0, 0.25, 0.42}
\definecolor{jazzberryjam}{rgb}{0.65, 0.04, 0.37}
\definecolor{lava}{rgb}{0.81, 0.06, 0.13}
\definecolor{royalblue}{rgb}{0.0, 0.14, 0.4}
\definecolor{prussianblue}{rgb}{0.0, 0.19, 0.33}
\definecolor{prune}{rgb}{0.44, 0.11, 0.11}
\definecolor{cerisepink}{rgb}{0.93, 0.23, 0.51}
\definecolor{oxfordblue}{rgb}{0.0, 0.13, 0.28}
\definecolor{crimsonglory}{rgb}{0.75, 0.0, 0.2}
\definecolor{fireenginered}{rgb}{0.81, 0.09, 0.13}

%============================
% Commands for inserting colored text.
\newcommand{\bluetext}[1]{\textcolor{darkblue}{#1}}
\newcommand{\redtext}[1]{\textcolor{jazzberryjam}{#1}}

%=================================================================================================
% Command for styling tabled row header (left, center or right)
% Usage example: \thead{<Header text 1>} & \thead{<Header 2>} & \thead{<Header 3>} & \thead{<Header 4>} 
\newcommand*{\thead}[1]{\multicolumn{1}{l}{\bfseries #1}}	

%--------------------------------------------------
% ==== Doc header and footer setup.               |
%-------------------------------------------------- 
\renewcommand{\thefootnote}{\fnsymbol{footnote}}
\pagestyle{fancyplain}
\fancyhf{}
%- Disable the horizontal ruler in the header section.
\renewcommand{\headrulewidth}{0pt}
\rfoot{\fancyplain{}{page \thepage\ of \pageref{LastPage}}}
\cfoot{{\tiny{\college { } - { } \semester} }}
\lfoot{{\tiny{ \coursenumber -\coursetitle} }}
%- TODO: move the header content here.
\fancyfoot[RO, LE] {{\tiny{page \thepage\ of \pageref{LastPage} }}}
\thispagestyle{plain}
%------------------------------------------------------------

\newcolumntype{L}[1]{>{\raggedright\arraybackslash}p{#1}}
\newcolumntype{C}[1]{>{\centering\arraybackslash}p{#1}}
\newcolumntype{R}[1]{>{\raggedleft\arraybackslash}p{#1}}

%-- Spacing commands ------ 
\newcommand{\vspbpara}{\vspace*{.09in}}    
\newcommand{\customvspace}{\vspace{.5cm}}    
\titlespacing{\section}{0pt}{12pt}{9pt}
%-----
\newcommand{\vtitlespacing}{\vskip 0.3cm}
\newcommand{\paragraphentry}[1]{\noindent \textbf{\Large \underline{#1}} }
   
%
%---> Genereate & inject metadata describing                |
%     the produced document                                 |
%--------------------------------------------------------------
%-- Set up the hyperref package.                              |
%-- Generate and inject metadate in the produced PDF document |
%------>------>------>------>------>------>------>------>-->---
 \hypersetup{pdfauthor={\instructor},%
    pdftitle={\coursenumber -- \coursetitle},%
    pdfsubject={\doctitle, Section \csection {} (\semester)},%
    pdfkeywords={\college,  \department},%
    pdfproducer={LaTeX},%
    pdfcreator={pdfLaTeX},
    bookmarks,
    bookmarksnumbered = true,
    bookmarksopen     = true,
    pdfpagelabels     = true,
    pdfstartview={XYZ null null 1.2}
}                                  %|
%------------------------------------------------------------

\topmargin      -60pt

%-----------------------------------------------------------
% Uncomment the following if you want to insert a watermark! 
%
%--> Watermark package settings: 
%\usepackage{draftwatermark}
%\SetWatermarkText{DRAFT}
%\SetWatermarkScale{0.5}
%\SetWatermarkColor[gray]{0.8}
%-------------------------------------------------

\begin{document} 
    
%-------------------------------------------------------------
%-- Make the header of the document                          |
%------>------>------>------>------>------>------>------>--> |
%--------------------------------------------------------------------------
%- The following produces the document header including the title.        |
%- The document header includes: the college/university name, faculty,    |
%  department, course number and title as well as the assignment/homework | 
%  title and due date.                                                    | 
%-------------------------------------------------------------------------|
%
\noindent % <-- need to have this first.
%
\begin{minipage}{.40\textwidth}
    {\color{darkred} \faSchool} { \textsc{\college}}{ } {\color{darkred} \faSchool}\\ 
    \small\textsc{ Faculty of Science \& Technology}\\%
    \small\textsc{Computer Science Technology}
\end{minipage}%
\hfill	
\begin{minipage}{0.60\textwidth}%
    \raggedleft%
    {\Large \textsc{\coursenumber { } \coursetitle}\par}
    \doublerule % insert a double rule.
    \textsc{Teacher}: \instructor\\
\end{minipage}%
\vspace{2.8cm}
{
    %--> Insert homework title and due date.
    \hrule\vspace{.2cm}
    \centering
    {\scshape 
        \Large \color{darkestblue}{\doctitle}{ }\textemdash{ }\small\bfseries\textsc{\semester}\par}
    \vspace{.3cm}    
}
{
    \hrule\vspace{.3cm}
    \centering  \small\duedate \\ 

}    
\vspace{3.5cm}


\hrule width0.3\textwidth
%---> Make the revision history.
\begin{versionhistory}
   \centering \vhEntry{1.0}{Feb 09, 2022}{S.R.}{Initial handout.}     
    \centering \vhEntry{1.1}{Feb 16, 2022}{S.R.}{Added the main character section.}     
    \centering \vhEntry{2.0}{Feb 19, 2022}{S.R.}{Modified and added new requirements in all sections.}     
\end{versionhistory}
\hrule
\vskip .3in
%
\tableofcontents

\clearpage
    
\section{Learning Objectives}      
\label{sec:objectives}    
\noindent 3D game design, HUD and in-game UI, working with multiple scenes, physics, Unity scripting with C\#, collision detection, using prefabs \& 3D arts, sound and visual effects as well as 3D animation, humanoid character and animation rigging. 

\section{Notes and Constraints}       
\label{sec:notes}    
\vspace{-.1cm}

\begin{minipage}{\linewidth}
\begin{bclogo}[couleur=gray!15, arrondi=0.1, logo=\bccrayon, ombre=true]{Note the following:}
   This is a thing to consider. \\
 Lorem ipsum dolor sit amet, consectetuer adipiscing elit. Etiam lobortis facilisis sem. Nullam nec mi et neque pharetra sollicitudin. Praesent imperdiet mi nec ante
\end{bclogo}
\end{minipage}


\begin{coloredPen}
    \item This assignment must be done individually.  
    \item Do not plagiarize.
    \item Do not do this.
    \item Do not do this.
    \item Do not do that. 
\end{coloredPen}
    
\section{Required Software and Tools }  
\label{sec:requiredsw}    
\begin{itemize}[itemsep=2pt,parsep=0pt,topsep=2pt,partopsep=2pt]
    %    \item[\color{darkblue}\faCoffee] Java 7 or 8 (32 or 64 bits)
    \item[\color{darkblue}\faLaptopCode] \textbf{Operating system:} \faWindows {} Windows  10,  \faLinux {} Linux, \textcolor{vanierred}{\textbf{or}} \faApple {} macOS 
    \item[\color{darkblue}\faCode] \textbf{IDE \& Game Engine:} \faUnity Unity \textcolor{vanierred}{2020.3 (LTS)} \textcolor{darkblue}{\&} Visual Studio \textcolor{vanierred}{2019} (Community Edition)
    \item [{\color{darkblue}\faChrome}] Web Browser: Google Chrome.   
    \item[{\color{darkblue} \faWpforms}] Markdown for writing documentation.
    \item[{\color{darkblue} \faGitSquare}] Distributed version control system.
    \item[{\color{darkblue} \faBitbucket}] Bitbucket: a web-based version control repository hosting service.
    \item[{\color{darkblue} \faTrello}] Trello: a Web-based project management system.
    
    \item[\color{darkblue}\faUsb]

    A storage medium (a USB flash memory or any online free storage service such as GDrive or OneDrive) for storing and backing up your files. 
\end{itemize}   

\section{Problem Statement}       
\label{sec:intro}    
\vspace{-.1cm}
\noindent In this assignment, you are required to design and implement...\\
\noindent \blindtext
\noindent Additional details and requirements are provided in the following sections.
          
    \section{Requirements}       
    \label{sec:requirements}    
    \vspace{-.1cm}
    \noindent You must be heedful of the requirements stated in the following sections. 
    
    \subsection{User Interfaces \& Game Menu}
    \label{sec:hudui}
    \noindent Your HUD, main menu and in-game panels (or other UI controls) must be implemented using the \href{https://docs.unity3d.com/Packages/com.unity.ugui@1.0/manual/index.html}{Unity UI toolkit}.
    \subsubsection{HUD}
    \label{sec:hudD}
    \noindent You must implement an in-game HUD that fulfills the following requirements:
    \begin{borderedsquare}
        \item Requirement 1.
        \item Requirement 2.
        \item Requirement 3.
        \item Requirement 4.
        \item etc...
    \end{borderedsquare}

\noindent {\color{rltblue}\large\bfseries\faBook} \space \textbf{Additional Resources}:

\begin{filledRightArrowList}
    \item Resource 1. 
    \item Resource 2. 
    \item Resource 3. 
    \item Resource 4. 
\end{filledRightArrowList}
    
    \subsubsection{Main Menu \& Player Feedback}
    \label{sec:mainmenu}
    \begin{filledstarlist}
        \item Implement this... 
        \item Implement this... 
        \item And implement that...         
    \end{filledstarlist}
    
    \subsection{Game Deployment}
    \label{sec:publish}
    You are required to publish your game implementation to...  
    
    \subsection{Game Implementation Requirements}
    \label{sec:implementation}
    Your game implementation must include the usage of:
    \begin{alphalist}
        \item 3D scenes.
        \item This requirement.
        \item And that requirement.
        \item Scripting:
        \begin{greenrectangles}
            \item Main character control, animation and movement
            \item Collision detection
        \end{greenrectangles}
        \item Animations: mainly for the main character. 
        \item Different visual effects (VFX), etc...
    \end{alphalist}    
\clearpage
\section{Evaluation Criteria}
\label{sec:evaluation}
\noindent Your assignment will be evaluated based on the following criteria:\\

\renewcommand{\arraystretch}{1.5} % this reduces the vertical spacing between rows    
\begin{tabular}{|p{14cm}|c|}
    \hline    
    \thead{\color{darkblue}  Criteria} |& \thead{\color{darkblue}Mark} \\ 
    \hline
    Game world design.    & 5\% \\      
    \hline
    Good programming and logging practices.    & 2\% \\      
    \hline 
    Relevance and accuracy of the source code documentation as instructed. & 3\% \\
    \hline
      Correctness and functionality of the implementation. & 60\% \\
    \hline
    Compliance of the implementation with the stated requirements. & 15\% \\
    \hline
    Programming style, etc... & 5\% \\
    \hline 
    Overall comprehension of the submitted source code. & 10\% \\	
    \hline 
    \textbf{Total}    & \textbf{100\%} \\
    \hline
\end{tabular}
    
%\clearpage
    
\section{What to Submit}
\label{sec:submit}
\noindent You must submit: 1) a PDF containing a list of references as instructed above; and 2) your Unity project. 
\begin{borderedsquare}		
    \item Remove the \textit{\color{violet}{Library, Temp and Builds folders}} from your Unity project.
    \item Create a folder and place in it your references document and your Unity project. 
    \item Compress the folder you just created and upload it to LÉA.
\end{borderedsquare}
    
\end{document} 